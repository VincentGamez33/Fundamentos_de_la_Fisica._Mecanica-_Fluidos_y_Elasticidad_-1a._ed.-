\chapter[Geometría Analítica]{GEOMETRÍA ANALÍTICA}
\startcontents
\printchaptertableofcontents

% La geometría analítica proporciona el vínculo entre las ideas geométricas y sus representaciones algebraicas. En física, este lenguaje resulta indispensable para describir trayectorias de partículas, la forma de superficies y las transformaciones de coordenadas que simplifican la resolución de problemas.

% En este apéndice se reúnen los conceptos fundamentales de geometría analítica que servirán como herramientas a lo largo del libro. No se pretende desarrollar una exposición exhaustiva del tema, sino presentar de manera clara y concisa los elementos necesarios para su aplicación en mecánica, fluidos y elasticidad.

% \section{Secciones cónicas}

% Comenzaremos con el estudio de las \textbf{secciones cónicas}, que aparecen en contextos como el análisis de órbitas planetarias o la propagación de ondas.

% \begin{definition}{}{conica}
%     Una \textbf{sección cónica} --o simplemente \textbf{cónica}-- es el conjunto de todos los puntos $P\,$ en un plano, tales que la razón de su distancia a un punto fijo $F\,$ --llamado \textbf{foco}-- y su distancia a una recta fija $L\,$ --llamada \textbf{directriz}-- es una constante positiva $e$. Esta razón constante se conoce como la \textbf{excentricidad} de la cónica:
%     $$e = \frac{d(P, F)}{d(P, L)}.$$
% \end{definition}

% Las secciones cónicas se clasifican en tres categorías, según su forma y propiedades. Estas se establecen de acuerdo con los valores de la excentricidad $e$.
% \begin{itemize}
%     \item Si $e<1$, la cónica se llama \textbf{elipse}.
%     \item Si $e=1$, la cónica se llama \textbf{parábola}.
%     \item Si $e>1$, la cónica se llama \textbf{hipérbola}.
% \end{itemize}

% \subsection{Parábola}

% \begin{definition}{}{parabola}
%     Una \textbf{parábola} es la sección cónica que se obtiene cuando la excentricidad es igual a 1.

%     Esto significa que una parábola es el lugar geométrico de todos los puntos $P$ del plano que \textbf{equidistan} de un punto fijo --el \textbf{foco} $F$-- y de una recta fija --la \textbf{directriz} $L$--.
%     $$d(P, F) = d(P, L).$$

%     Para encontrar su ecuación canónica, se sitúa el \textbf{vértice} --el punto de la parábola más cercano a la directriz-- en el origen $(0,0)$ y se alinea su eje de simetría con el eje $y$. En esta configuración, el foco se localiza en el punto $F(0, p)$ y la directriz es la recta horizontal $L$ con ecuación $y = -p$.

%     La distancia de un punto cualquiera $P(x, y)$ de la parábola al foco es $d(P, F) = \sqrt{x^2 + (y-p)^2}$, y su distancia a la directriz es $d(P, L) = |y+p|$. Al igualar ambas distancias y simplificar la expresión, se obtiene la \textbf{ecuación canónica de la parábola} con eje vertical:
%     $$x^2 = 4py.$$
    
%     El valor de $p\,$ representa la distancia del vértice al foco. El signo de $p$ determina la orientación de la parábola: si $p>0$, se abre hacia arriba; si $p<0$, se abre hacia abajo.
% \end{definition}

La geometría analítica proporciona el vínculo entre las ideas geométricas y sus representaciones algebraicas. En física, este lenguaje resulta indispensable para describir trayectorias de partículas, la forma de superficies y las transformaciones de coordenadas que simplifican la resolución de problemas.

En este apéndice se reúnen los conceptos fundamentales de geometría analítica que servirán como herramientas a lo largo del libro. No se pretende desarrollar una exposición exhaustiva del tema, sino presentar de manera clara y concisa los elementos necesarios para su aplicación en mecánica, fluidos y elasticidad.

\section{Secciones cónicas}

Comenzaremos con el estudio de las \textbf{secciones cónicas}, que aparecen en contextos como el análisis de órbitas planetarias, la trayectoria de proyectiles o la propagación de ondas.

\begin{definition}{}{conica}
    Una \textbf{sección cónica}, o simplemente \textbf{cónica}, es el conjunto de todos los puntos $P$ en un plano tales que la razón de su distancia a un punto fijo $F$ ---llamado \textbf{foco}--- y su distancia a una recta fija $L$ ---llamada \textbf{directriz}--- es una constante positiva $e$. Esta razón constante se conoce como la \textbf{excentricidad} de la cónica:
    $$e = \frac{d(P, F)}{d(P, L)}.$$
\end{definition}

Las cónicas se clasifican en tres categorías de acuerdo con el valor de la excentricidad $e$:
\begin{itemize}
    \item Si $e<1$, la cónica se llama \textbf{elipse}.
    \item Si $e=1$, la cónica se llama \textbf{parábola}.
    \item Si $e>1$, la cónica se llama \textbf{hipérbola}.
\end{itemize}

En las siguientes subsecciones estudiaremos cada caso por separado, comenzando con la parábola.

\subsection{Parábola}

\begin{definition}{}{parabola}
    Una \textbf{parábola} es la sección cónica que se obtiene cuando la excentricidad es igual a 1. En otras palabras, la parábola es el lugar geométrico de todos los puntos $P$ del plano que \textbf{equidistan} de un punto fijo ---el \textbf{foco} $F$--- y de una recta fija ---la \textbf{directriz} $L$---:
    $$d(P, F) = d(P, L).$$

    Para deducir su ecuación canónica, se coloca el \textbf{vértice} ---el punto de la parábola más cercano a la directriz--- en el origen $(0,0)$ y se alinea su eje de simetría con el eje $y$. En esta disposición, el foco se ubica en $F(0,p)$ y la directriz es la recta horizontal $y=-p$.

    La distancia de un punto cualquiera $P(x,y)$ al foco es $d(P,F)=\sqrt{x^2+(y-p)^2}$, mientras que su distancia a la directriz es $d(P,L)=|y+p|$. Al igualar ambas expresiones y simplificar, se obtiene la \textbf{ecuación canónica de la parábola} con eje vertical:
    $$x^2 = 4py.$$

    El parámetro $p$ representa la distancia del vértice al foco (y también del vértice a la directriz). Su signo determina la orientación de la parábola: si $p>0$, se abre hacia arriba; si $p<0$, se abre hacia abajo.
\end{definition}

Las parábolas aparecen en numerosos fenómenos físicos: describen la trayectoria de un proyectil bajo un campo gravitatorio uniforme (sin resistencia del aire) y son la base en el diseño de superficies reflectoras, como espejos parabólicos y antenas.


\subsection{Elipse}
\subsection{Hipérbola}

\section{Otros sistemas coordenados}

\subsection{Coordenadas polares}
\subsection{Coordenadas cilíndricas}
\subsection{Coordenadas esfércias}

\section{Transformación de coordenadas}

\subsection{Traslación de ejes}
\subsection{Rotación de ejes}

\section{Curvas en el plano}

\subsection{Tangente}
\subsection{Normal}
\subsection{Radio de curvatura}
En el estudio de las curvas planas no basta con conocer su ecuación cartesiana. 
También es de interés determinar cómo se dobla la curva en cada uno de sus puntos. 
Esta propiedad recibe el nombre de \textbf{curvatura} y mide el grado en que la tangente cambia de dirección al desplazarse a lo largo de la curva.

En un punto dado $P$, existe siempre una circunferencia que se ajusta de manera más exacta a la curva: además de pasar por el punto y compartir la recta tangente, coincide con la curva en la forma en que ésta se curva. A dicha circunferencia se le denomina \textbf{círculo osculador}, y el radio de esta circunferencia se llama \textbf{radio de curvatura} de la curva en $P$.

Para calcular este radio a partir de la ecuación de la curva, es necesario relacionar el giro de la tangente con las derivadas de la función: la primera derivada $y'=\dfrac{dy}{dx}$ proporciona la pendiente de la tangente, mientras que la segunda derivada $y''=\dfrac{d^2y}{dx^2}$ describe la rapidez con que dicha pendiente varía, es decir, el cambio en la dirección de la tangente.

Estas observaciones conducen a la siguiente proposición, que da una fórmula práctica del radio de curvatura en coordenadas cartesianas.

\begin{proposition}{}{radio_de_curvatura}
    Sea $y=f(x)$ una curva plana con $f\in C^2$ en un entorno de $x_0$. Denotemos $P=(x_0,y_0)\,$ con $y_0=f(x_0)$. Si la tangente está bien definida en $P$, entonces el radio de curvatura en $P\,$ viene dado por
    \[R=\dfrac{\left[1+\left(\dfrac{dy}{dx}\right)^2\right]^{3/2}}{\left|\dfrac{d^2y}{dx^2}\right|}.\]
    \begin{demostracion}
        Sea una circunferencia de radio $R$ y centro $(h,k)$ tangente a la curva en $P$. La ecuación de la circunferencia es
        $$ (x-h)^2+(y-k)^2=R^2,$$
        donde $h,k,R \in \RR$ son constantes. Derivando implícitamente respecto a $x$ y evaluando en $P$ obtenemos
        \begin{align*}
            \frac{d}{dx} [(x-h)^2 + (y-k)^2] &= \frac{d}{dx}R^2 \\
            2(x-h) + 2(y-k)\frac{dy}{dx} &= 0 &&\text{(multiplicando por $\tfrac{1}{2}$)}\\
            (x-h) + (y-k)\frac{dy}{dx} &= 0.
        \end{align*}
        Ahora, derivando implícitamente respecto a $x$ la expresión anterior
        \begin{align*}
            \frac{d}{dx} (x-h) + (y-k)\frac{dy}{dx} &= \frac{d}{dx} 0 \\
            1 + (y-k)\frac{d^2y}{dx^2} + \left(\frac{dy}{dx}\right)\frac{dy}{dx} &= 0 \\
            1 + (y-k)\frac{d^2y}{dx^2} + \left(\frac{dy}{dx}\right)^2 &= 0.
        \end{align*}
        Obteniendo el siguiente sistema de ecuaciones
        \begin{equation}\label{eq:circunferencia}
            (x-h)^2 + (y-k)^2 = R^2,
        \end{equation}
        \begin{equation}\label{eq:radio_1}
            (x-h) + (y-k)\frac{dy}{dx} = 0,
        \end{equation}
        \begin{equation}\label{eq:radio_2}
            1 + (y-k)\frac{d^2y}{dx^2} + \left(\frac{dy}{dx}\right)^2 = 0.
        \end{equation}
        De la ecuación \eqref{eq:radio_2} se sigue
        \begin{align*}
            (y-k)\frac{d^2y}{dx^2} &= -\left[1 + \left(\frac{dy}{dx}\right)^2\right] \\
            y-k &= -\frac{1 + \left(\dfrac{dy}{dx}\right)^2}{\dfrac{d^2y}{dx^2}},
        \end{align*}
        y de la ecuación \eqref{eq:radio_1} obtenemos
        $$x-h = -(y-k)\frac{dy}{dx}$$
        sustituyendo el valor de $y-k$ en lo anterior se tiene
        $$x-h = \frac{1 + \left(\dfrac{dy}{dx}\right)^2}{\dfrac{d^2y}{dx^2}}\frac{dy}{dx}.$$
        Luego, sustituyendo los valores de $x-h$ y $y-k$ en la ecuación \eqref{eq:circunferencia}
        \begin{align*}
            \left[\frac{1 + \left(\dfrac{dy}{dx}\right)^2}{\dfrac{d^2y}{dx^2}}\frac{dy}{dx}\right]^2 + \left[-\frac{1 + \left(\dfrac{dy}{dx}\right)^2}{\dfrac{d^2y}{dx^2}}\right]^2 &= R^2 \\
            \left[\frac{1 + \left(\dfrac{dy}{dx}\right)^2}{\dfrac{d^2y}{dx^2}}\right]^2\left[1+\left(\frac{dy}{dx}\right)^2\right] &= R^2 \\
            \frac{\left(1 + \left(\dfrac{dy}{dx}\right)^2\right)^3}{\left(\dfrac{d^2y}{dx^2}\right)^2} &= R^2 \\
            R &= \sqrt{\frac{\left(1 + \left(\dfrac{dy}{dx}\right)^2\right)^3}{\left(\dfrac{d^2y}{dx^2}\right)^2}} \\
            R &= \frac{\left(1 + \left(\dfrac{dy}{dx}\right)^2\right)^{3/2}}{\left|\dfrac{d^2y}{dx^2}\right|}.
        \end{align*}
        Por lo tanto, queda demostrado que $R=\dfrac{\left[1+\left(\dfrac{dy}{dx}\right)^2\right]^{3/2}}{\left|\dfrac{d^2y}{dx^2}\right|}$.
    \end{demostracion}
\end{proposition}
\begin{demostracion}
    Sabemos por identidades trigonométricas que $ds^2 = dx^2 + dy^2$, entonces
    \begin{align*}
        ds^2 &= dx^2 + dy^2 \\
        ds^2 &= dx^2\left(1 + \frac{dy^2}{dx^2}\right) \\
        ds &= \sqrt{dx^2\left(1 + \frac{dy^2}{dx^2}\right)} \\
        ds &= \left[1 + \left(\frac{dy}{dx}\right)^2\right]^{1/2} dx \label{eq:ds}.
    \end{align*}
    Además, sabemos que por los cursos de Geometría Analítica y Cálculo I que la longitud de arco de una  circunferencia está dada por
    \[s = R\theta\]
    lo cual implica que
    \[ds = R\,d\theta,\]
    o bien
    \[R = \frac{ds}{d\theta}.\]
    Luego, para $\theta$ sabemos que
    \begin{align*}
        \tan \theta &= \frac{dy}{dx} \\
        \theta &= \arctan \left(\frac{dy}{dx}\right).
    \end{align*}
    Sea $v = \dfrac{dy}{dx}$, entonces derivando implícitamente y aplicando regla de cadena tenemos
    \begin{align*}
        \frac{d\theta}{dx} &= \frac{\frac{dv}{dx}}{1+v^2} \\
        d\theta &= \frac{\frac{d}{dx}\left(\frac{dy}{dx}\right)}{1 + \left(\frac{dy}{dx}\right)^2} dx \\
        d\theta &= \frac{\frac{d^2y}{dx^2}}{1 + \left(\frac{dy}{dx}\right)^2} dx.
    \end{align*}
    Sustituyendo el valor de $d\theta$ y $ds$ en $R$ se sigue
    \begin{align*}
        R &= \frac{\left[1 + \left(\frac{dy}{dx}\right)^2\right]^{1/2} dx}{\frac{\frac{d^2y}{dx^2}}{1 + \left(\frac{dy}{dx}\right)^2} dx} \\
        R &= \frac{\left[1 + \left(\frac{dy}{dx}\right)^2\right]^{3/2}}{\frac{d^2y}{dx^2}}.
    \end{align*}
    Sabemos que $\frac{d^2y}{dx^2}$ nos da la concavidad de la función, para simplicar el análisis aplicaremos valor absoluto. Por lo tanto, queda demostrado que el radio de curvatura es de la forma
    \[R=\dfrac{\left[1+\left(\dfrac{dy}{dx}\right)^2\right]^{3/2}}{\left|\dfrac{d^2y}{dx^2}\right|}.\]
\end{demostracion}

\cleardoublepage


 % Posteriormente se introducirán \textbf{otros sistemas de coordenadas}, útiles para aprovechar simetrías geométricas en problemas físicos. A continuación se revisarán las \textbf{transformaciones de coordenadas}, necesarias para simplificar expresiones matemáticas mediante traslaciones y rotaciones de ejes. Finalmente, se estudiarán las \textbf{curvas en el plano}, con énfasis en la definición de tangente, normal y radio de curvatura, herramientas básicas para la descripción local del movimiento.